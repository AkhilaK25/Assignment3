\documentclass[journal,12pt,twocolumn]{IEEEtran}
\usepackage[utf8]{inputenc}
\usepackage{graphicx}
\usepackage{amsmath}
\usepackage{mathrsfs}
\usepackage{txfonts}
\usepackage{stfloats}
\usepackage{bm}
\usepackage{cite}
\usepackage{cases}
\usepackage{subfig}
\usepackage{amsfonts}
\usepackage{amssymb}
\usepackage{enumitem}
\usepackage{mathtools}
\usepackage{tikz}
\usepackage{circuitikz}
\usepackage{verbatim}
\usepackage[breaklinks=false,hidelinks]{hyperref}
\usepackage{listings}
\usepackage{calc}
\usepackage{float}
\usepackage{longtable}
\usepackage{multirow}
\usepackage{multicol}
\usepackage{color}
\usepackage{array}
\usepackage{hhline}
\usepackage{ifthen}
\usepackage{chngcntr}


\bibliographystyle{IEEEtran}
%\bibliographystyle{ieeetr}
\def\inputGnumericTable{}
\let\vec\mathbf
\providecommand{\pr}[1]{\ensuremath{\Pr\left(#1\right)}}
\providecommand{\sbrak}[1]{\ensuremath{{}\left[#1\right]}}
\providecommand{\lsbrak}[1]{\ensuremath{{}\left[#1\right.}}
\providecommand{\rsbrak}[1]{\ensuremath{{}\left.#1\right]}}
\providecommand{\brak}[1]{\ensuremath{\left(#1\right)}}
\providecommand{\lbrak}[1]{\ensuremath{\left(#1\right.}}
\providecommand{\rbrak}[1]{\ensuremath{\left.#1\right)}}
\providecommand{\cbrak}[1]{\ensuremath{\left\{#1\right\}}}
\providecommand{\lcbrak}[1]{\ensuremath{\left\{#1\right.}}
\providecommand{\rcbrak}[1]{\ensuremath{\left.#1\right\}}}
\providecommand{\res}[1]{\Res\displaylimits_{#1}}
\newcommand{\myvec}[1]{\ensuremath{\begin{pmatrix}#1\end{pmatrix}}}
\newcommand{\mydet}[1]{\ensuremath{\begin{vmatrix}#1\end{vmatrix}}}


\newcommand{\question}{\noindent \textbf{Question: }}
\newcommand{\solution}{\noindent \textbf{Solution: }}
\newcommand{\note}{\noindent \textbf{Note: }}

\begin{document}
\title{ASSIGNMENT 3}
\author{AKHILA, CS21BTECH11031}

\maketitle
\question
A box contains 3 blue, 2 white and 4 red marbles.If a marble is drawn at random from the box, what is the probability that it will be
    \begin{enumerate}[label=(\roman{enumi})]
		\item white?
		\item blue?
		\item red?
    \end{enumerate}
\solution
Given there are 2 white, 3 blue and 4 red marbles.\\ 
\begin{table}[H]
\def\ifundefined#1{\expandafter\ifx\csname#1\endcsname\relax}

\ifundefined{inputGnumericTable}

	\def\gnumericTableEnd{\end{document}}

	\documentclass[12pt%
			  %,landscape%
                    ]{report}
       \usepackage[latin1]{inputenc}
       \usepackage{fullpage}
       \usepackage{color}
       \usepackage{array}
       \usepackage{longtable}
       \usepackage{calc}
       \usepackage{multirow}
       \usepackage{hhline}
       \usepackage{ifthen}

\begin{document}

\else

    \def\gnumericTableEnd{}
\fi
\providecommand{\gnumericmathit}[1]{#1} 

\providecommand{\gnumericPB}[1]%
{\let\gnumericTemp=\\#1\let\\=\gnumericTemp\hspace{0pt}}
 \ifundefined{gnumericTableWidthDefined}
        \newlength{\gnumericTableWidth}
        \newlength{\gnumericTableWidthComplete}
        \newlength{\gnumericMultiRowLength}
        \global\def\gnumericTableWidthDefined{}
 \fi

 \ifthenelse{\isundefined{\languageshorthands}}{}{\languageshorthands{english}}
\providecommand\gnumbox{\makebox[0pt]}
\setlength{\bigstrutjot}{\jot}
\setlength{\extrarowheight}{\doublerulesep}

\setlongtables

\setlength\gnumericTableWidth{%
	125pt+%
	225pt+%
0pt}
\def\gumericNumCols{2}
\setlength\gnumericTableWidthComplete{\gnumericTableWidth+%
         \tabcolsep*\gumericNumCols*2+\arrayrulewidth*\gumericNumCols}
\ifthenelse{\lengthtest{\gnumericTableWidthComplete > \linewidth}}%
         {\def\gnumericScale{1*\ratio{\linewidth-%
                        \tabcolsep*\gumericNumCols*2-%
                        \arrayrulewidth*\gumericNumCols}%
{\gnumericTableWidth}}}%
{\def\gnumericScale{1}}


\ifthenelse{\isundefined{\gnumericColA}}{\newlength{\gnumericColA}}{}\settowidth{\gnumericColA}{\begin{tabular}{@{}p{125pt*\gnumericScale}@{}}x\end{tabular}}
\ifthenelse{\isundefined{\gnumericColB}}{\newlength{\gnumericColB}}{}\settowidth{\gnumericColB}{\begin{tabular}{@{}p{225pt*\gnumericScale}@{}}x\end{tabular}}


  \begin{center}
\begin{tabular}[c]{%
	b{\gnumericColA}%
	b{\gnumericColB}%
	}


\hhline{|-|-|}
	 \multicolumn{1}{|p{\gnumericColA}|}%
	{\gnumericPB{\centering}\gnumbox{\textbf{Event}}}
	&\multicolumn{1}{p{\gnumericColB}|}%
	{\gnumericPB{\centering}\gnumbox{\textbf{Description}}}
\\

\hhline{|--|}
	 \multicolumn{1}{|p{\gnumericColA}|}%
	{\gnumericPB{\centering}\gnumbox{X=0}}
	&\multicolumn{1}{p{\gnumericColB}|}%
	{\gnumericPB{\centering}\gnumbox{white marble is drawn}}
\\
\hhline{|--|}
	 \multicolumn{1}{|p{\gnumericColA}|}%
	{\gnumericPB{\centering}\gnumbox{X=1}}
	&\multicolumn{1}{p{\gnumericColB}|}%
	{\gnumericPB{\centering}\gnumbox{blue marble is drawn}}
\\
\hhline{|--|}
	 \multicolumn{1}{|p{\gnumericColA}|}%
	{\gnumericPB{\centering}\gnumbox{X=2}}
	&\multicolumn{1}{p{\gnumericColB}|}%
	{\gnumericPB{\centering}\gnumbox{red marble is drawn}}
\\

\hhline{|-|-|}
\end{tabular}
 \end{center} 
\ifthenelse{\isundefined{\languageshorthands}}{}{\languageshorthands{\languagename}}
\gnumericTableEnd
	\caption{Random variable and Event distribution}
	\label{tab1}
\end{table}
\begin{enumerate}[label=(\roman{enumi})]
      \item The probability that a marble is drawn
at random from the box is white
\begin{align}
	 &\pr{W} = \frac{\text{Number of white marbles}}{\text{Total number of marbles}}\\
	 &=\frac{2}{9}\\
	 &=0.222
\end{align}
      \item The probability that a marble is drawn
at random from the box is blue
\begin{align}
	 &\pr{B} = \frac{\text{Number of blue marbles}}{\text{Total number of marbles}}\\
	 &=\frac{3}{9}\\
	 &=\frac{1}{3}\\
	 &=0.333
\end{align}
      \item The probability that a marble is drawn
at random from the box is red
\begin{align}
	 &\pr{R} = \frac{\text{Number of red marbles}}{\text{Total number of marbles}}\\
	 &=\frac{4}{9}\\
	 &=0.444
\end{align}
\note  Since the events are mutually exclusive and exhaustive, the probability that the ball drawn is red can also be calculated as
\begin{align}
	&\pr{W}+\pr{B}+\pr{R}=1\\
    &\pr{R}=1-\pr{W}-\pr{B}\\
    &= 1-\frac{2}{9}-\frac{1}{3}\\
    &=\frac{4}{9}\\
    &=0.444
    \end{align}
\end{enumerate}
\end{document}